\documentclass[12pt]{article}

\usepackage{amsthm}
\usepackage{amssymb}
\usepackage{amsmath}
\usepackage{chngcntr}
\usepackage{graphicx}
\usepackage{microtype}
\usepackage{enumitem}
\usepackage{hyperref}

\newcounter{axmmain}
\newtheorem{axminner}{Axiom}[axmmain]
\makeatletter
\renewcommand{\theaxminner}{%
  \arabic{axmmain}.\@arabic{\numexpr\value{axminner}-1\relax}%
}
\makeatother
\newenvironment{axm}
 {\stepcounter{axmmain}\axminner}
 {\endaxminner}
\newenvironment{axm*}
 {\axminner}
 {\endaxminner}

\newcommand*{\oldneg}{\mathord{\sim}}

\newcounter{prmc}
\newcounter{dfnc}
\newcounter{thmc}
\newcounter{exc}

\newtheorem{prm}[prmc]{Primitive Notion}
\newtheorem{dfn}[dfnc]{Definition}
\newtheorem{thm}[thmc]{Theorem}
\newtheorem{cor}[thmc]{Corollary}
\newtheorem{lem}[thmc]{Lemma}
\newtheorem{ex}[exc]{Example}

\title{Proofs}
\author{Joshua Meyers}

\begin{document}

\date{\today}
\maketitle

This document is intended to state the ``rules'' of mathematical
proofs for Russell Dale.

\section{Preliminaries}

I will not define ``set'' or ``element'', as these are primitive
notions.  I will also assume predicate logic, since I know you know
that.

\begin{prm}
  Set
\end{prm}

\begin{prm}
  Element.  For a set $S$, $x\in S$ denotes that $x$ is an element of
  $S$.
\end{prm}

For convenience we define the negation of $\in$:

\begin{dfn}
  For a set $S$, $x\not\in S$ iff $\oldneg x\in S$
\end{dfn}

\section{The Axiom of Extention}

Here is our first axiom:

\begin{axm}[Axiom of extension]
  For sets $S$ and $T$, $S=T$ iff $S$ and $T$ have all the same
  elements.
\end{axm}

The axiom of extension is basically an admission that ``set'' is a
concept of universal that is purely extensional.  Version
\ref{axm:ext} makes this clear:

\begin{axm*}[Axiom of extension]\label{axm:ext}
  For sets $S$ and $T$, $S=T$ iff $S$ and $T$ have the same extension.
\end{axm*}

We can also rephrase the axiom of extension in terms of the $\in$
relation.  This makes precise what we mean by ``having all the same
elements''.

\begin{axm*}[Axiom of extension]\label{axm:ext-bicond}
  For sets $A$ and $B$, $A=B$ iff $(\forall x)(x\in A \Leftrightarrow
  x\in B)$.
\end{axm*}

This form is more useful because it only uses our primitive notions
and logic.  It is useful for proving that two sets are equal.  To
prove that $A=B$, we must show that the property $x\in A$ is
equivalent to the property $x\in B$.

\begin{ex}
  Let $A=\{2k+1 | k \in \mathbb{Z}\}$ and $B=\{2k-1 |
  k\in\mathbb{Z}\}$.  We show that $A=B$ by showing that $x\in A
  \Leftrightarrow x\in B$.  First suppose that $x\in A$.  Then
  $\exists k$ such that $x=2k+1$.  We use existential instantiation to
  remove the quantifier and obtain $x=2k+1$.  Now by algebra we obtain
  $x=2(k+1)-1$, so by existential generalization, $\exists j$, namely
  $k+1$, such that $x=2j-1$, so $x\in B$.  Thus $x\in A \Rightarrow
  x\in B$.  By a similar argument, we can show that $x\in B
  \Rightarrow x\in A$.  Thus $x\in A \Leftrightarrow x\in B$.  By the
  axiom of extension, $A=B$.
\end{ex}

Now we define subsets.  We denote ``$A$ is a subset of $B$'' by
$A\subseteq B$ and ``$A$ is not a subset of $B$'' by $A\not\subseteq
B$.  We denote ``$A$ is a superset of $B$'' by $A\supseteq B$ and
``$A$ is not a superset of $B$'' by $A\not\supseteq B$,

\begin{dfn}\label{dfn:subset}
  For sets $A$ and $B$,
  \begin{enumerate}[label=\alph*)]
  \item $A\subseteq B$ iff $(\forall x)(x\in A \Rightarrow x\in B)$
  \item $A\not\subseteq B$ iff $\oldneg A\subseteq B$
  \item $A\supseteq B$ iff $B\subseteq A$
  \item $A\not\supseteq B$ iff $\oldneg A\supseteq B$
  \end{enumerate}
\end{dfn}

The usual approach to prove that $A\subseteq B$ is to use a direct
proof to show that for an arbitrary $x$, $x\in A \Rightarrow x\in B$.
So we start by supposing that $x\in A$, and we show from that that
$x\in B$.  We conclude that $(\forall x)(x\in A \Rightarrow x\in B)$,
which is the definition of $A\subseteq B$.

\begin{ex}
  Let $A$ be the set of all multiples of $4$, and $B$ be the set of
  all multiples of $2$.  We want to show that $A\subseteq B$.  Suppose
  $x\in A$.  Then $x$ is a multiple of $4$, so there is an integer $k$
  such that $x=4k$.  But then $x=2(2k)$, so $x$ is a multiple of $2$
  and $x\in B$.  So $x\in A \Rightarrow x\in B$, which is equivalent
  to $A\subseteq B$.
\end{ex}


A useful method for proving that two sets are equal is to prove that
they are subsets of each other.  This method is based on the truth of
the following theorem:

\begin{thm}\label{thm:eq}
  For sets $A$ and $B$, $A=B$ iff $A\subseteq B$ and $B\subseteq A$.
\end{thm}

\begin{proof}
  \begin{align*}
    A=B
    &\Leftrightarrow (\forall x)(x\in A \Leftrightarrow x\in B)\\
    &\Leftrightarrow (\forall x)((x\in A \Rightarrow x\in B) \wedge (x\in B \Rightarrow x\in A)) \\
    &\Leftrightarrow ((\forall x)(x\in A \Rightarrow x\in B) \wedge (\forall x)(x\in B \Rightarrow x\in A)) \\
    &\Leftrightarrow ((A\subseteq B) \wedge (B\subseteq A))
  \end{align*}
  The first equivalence comes from Axiom \ref{axm:ext-bicond}, the
  second and third come from logic, and the fourth comes from
  Def. \ref{dfn:subset}.
\end{proof}
   
So it seems that we are building a correspondence between set theory
and logic, so that we can prove things about sets using facts from
logic.  Axiom \ref{axm:ext-bicond} expresses equality of sets in terms
of equivalence of predicates.  Def. \ref{dfn:subset} defines the
subset relation in terms of the material conditional.  In general,
statements about sets reduce to statements about logic.

\section{The ``Axiom'' of Comprehension}

Now we need an axiom that can help us make sets.

\begin{axm}[``Axiom'' of comprehension]\label{axm:comp}
  For a predicate $P$, there exists a set $S$ such that $x\in S$ iff
  $Px$.  The set $S$ is then denoted $\{x|Px\}$.
\end{axm}

This ``axiom'' is not really an axiom, because it implies Russell's
Paradox.  Still, if we avoid Russell's paradox, it is sufficient for
most purposes.  We will use it for now and bracket the issue that it
does not really work.

Before Axiom \ref{axm:comp}, we could go from a set $S$ to the
predicate $x\in S$, which is easier to work with.  Now that we have
Axiom \ref{axm:comp}, we can go the other way, from a predicate $P$ to
the set $\{x|Px\}$.  So now we can go both ways, from sets to
predicates and from predicates to sets.  Sets and predicates arguably
contain the same information.  In the current mathematical jargon, to
change a data type while preserving information is to ``induce''.  So,
accordingly, we can say that a set $S$ \textit{induces} the predicate
$x\in S$ and a predicate $P$ \textit{induces} the set $\{x|Px\}$.

Now we define the null set, using the ``axiom'' of comprehension.

\begin{dfn}
  $\emptyset=\{x|x\neq x\}$
\end{dfn}

We have just defined the empty set as the set induced by the predicate
$x\neq x$, which is never satisfied.  Since the predicate is never
satisfied, the set it induces has no elements.  But we could have used
a different contradictory predicate, such as ``1=0''. (This doesn't
mention $x$, but it doesn't have to.  It is possible for a predicate
to ignore its argument completely.)  Why didn't we define the null
set as $\{x|1=0\}$?  Would it have come out any different?  The
following theorem answers this question:

\begin{thm}
  The null set is unique.  In other words, if $P$ is a contradictory
  predicate, then $\{x|Px\}=\emptyset$.
\end{thm}

\begin{proof}
  Suppose $P$ is a contradictory predicate.  We will use Theorem
  \ref{thm:eq} to prove that $\{x|Px\}$ and $\emptyset$ are equal by
  proving first that they are subsets of each other.  First we prove
  that $\{x|Px\}\subseteq\emptyset$.  Suppose that
  $\{x|Px\}\not\subseteq\emptyset$.  Then $\exists x$ such that
  $x\in\{x|Px\}$ and $x\not\in\emptyset$, which implies that $Px$, a
  contradiction.  So $\{x|Px\}\subseteq\emptyset$.  We can prove that
  $\emptyset\subseteq\{x|Px\}$ by a similar argument.  Thus
  $\{x|Px\}=\emptyset$.
\end{proof}

This is a great illustration of the extensionality of the concept of
set.  Noah Schweber
\href{https://math.stackexchange.com/questions/2296639/is-the-empty-set-or-any-analogy-ever-non-unique#comment4724973_2296639}{writes
  in a comment} on StackExchange:

\begin{quote}

  I think proving that the emptyset is unique is a good piece towards
  demonstrating the extensionality, rather than intensionality, of set
  theory. The emptyset is possibly the most natural set given to lots
  of different intensional definitions: the set of counterexamples to
  Fermat and the set of primes with rational square roots are each the
  empty set, but clearly are different definitions. And for whatever
  reason, it's the set which seems to cause the most trouble in this
  regard. As trivial as it is, using the axiom of extensionality here
  plants the seed of extensional thinking.

\end{quote}

\section{Some More Theorems}

Now I will state some more theorems without proof that can be proved
with this same method: convert the statement about sets to a statement
about the induced predicates and then use logic.

\begin{thm}
  For a set $A$,
  \begin{enumerate}[label=\alph*)]
  \item $\emptyset\subseteq A$
  \item $A\subseteq A$
  \end{enumerate}
\end{thm}

\begin{thm}\label{thm:poset}
  The subset relation is a poset.  In other words, for sets $A$, $B$,
  and $C$,
  \begin{enumerate}[label=\alph*)]
  \item $A\subseteq A$ (reflexivity)
  \item $(A\subseteq B \wedge B\subseteq A)\Rightarrow A=B$ (antisymmetry)
  \item $(A\subseteq B \wedge B\subseteq C)\Rightarrow A\subseteq C$ (transitivity)
  \end{enumerate}
\end{thm}

We now define intersection and union in terms of logical operations on
the induced predicates.

\begin{dfn}
  For sets $A$ and $B$, the union $A\cup B = \{x|x\in A \vee x\in B\}$
  and the intersection $A\cap B = \{x|x\in A \wedge x\in B\}$.
\end{dfn}

The following set-theoretic identities can all be proved from the
corresponding logical laws obtained from replacing $\cup$ with $\vee$
and $\cap$ with $\wedge$.

\begin{thm}\label{thm:op}
  The operations $\cup$ and $\cap$ are commutative, associative, and
  idempotent, and each distributes over the other.  In other words,
  for sets $A$, $B$, and $C$,
  \begin{enumerate}[label=\alph*)]
  \item $A\cup B=B\cup A$ and $A\cap B=B\cap A$
  \item $A\cup (B\cup C)=(A\cup B)\cup C$ and $A\cap (B\cap C)=(A\cap B)\cap C$
  \item $A\cup A=A$ and $A\cap A=A$
  \item $A\cup(B\cap C)=(A\cup B)\cap(A\cup C)$ and $A\cap(B\cup C)=(A\cap
    B)\cup(A\cap C)$
  \end{enumerate}
\end{thm}

\begin{proof}
  We will prove just the distributivity of union over intersection
  (the first part of item d)) and leave the rest unproved.  We prove
  the equality of $A\cup(B\cap C)$ and $(A\cup B)\cap(A\cup C)$
  through \ref{axm:ext-bicond}.
  \begin{align*}
    x\in A\cup(B\cap C)
    &\Leftrightarrow (x\in A)\vee(x\in B\cap C) \\
    &\Leftrightarrow (x\in A)\vee(x\in B\wedge x\in C) \\
    &\Leftrightarrow (x\in A\vee x\in B)\wedge (x\in A\vee x\in C)\\
    &\Leftrightarrow (x\in A\cup B)\wedge (x\in A\cup C)\\
    &\Leftrightarrow x\in (A\cup B)\cap(A\cup C)
  \end{align*}
\end{proof}

So far we have a set-theoretic relation $\subseteq$, which makes sets
into a poset (cf. Thm. \ref{thm:poset}, and two set-theoretic
operators, $\cup$ and $\cap$.  It turns out that these operators have
a special place with respect to the poset, as shown in the following
theorem.

\begin{thm}
  For sets $A$ and $B$,
  \begin{enumerate}[label=\alph*)]
    \item $A\cup B$ is the \textit{least upper bound} of $A$ and $B$
      with respect to the $\subseteq$ ordering.  In other words,
      $A\cup B\supseteq A, B$ (it is \textit{an} upper bound of $A$
      and $B$) and for any set $C$, $C\supseteq A, B$ implies that
      $C\supset eq A\cup B$ (it is a subset of any upper bound of $A$
      and $B$). (Notice that we have used $\supseteq$, not $\subseteq$
      here, so that it looks similar to part b).)
    \item $A\cap B$ is the \textit{greatest lower bound} of $A$ and
      $B$ with respect to the $\subseteq$ ordering.  In other words,
      $A\cap B\subseteq A,B$ and for any set $C$, $C\subseteq A,B$
      implies that $A\cap B \subseteq C$.
  \end{enumerate}
\end{thm}

\begin{proof}
  \begin{enumerate}[label=\alph*)]
    \item First we will prove that $A\cup B\supseteq A,B$.  If $x\in
      A$, then by disjunction introduction $x\in A \vee x\in B$, so
      $x\in A\cup B$.  Thus $A\subseteq A\cup B$.  By a similar
      argument, $B\subseteq A\cup B$.  Second we will prove that for
      any set $C$ such that $A,B\subseteq C$, $A\cup B\subseteq C$.
      Suppose that $C$ is a set and $A,B\subseteq C$.  If $x\in A\cup
      B$, then $x\in A \vee x\in B$.  Either disjunct of this last
      statement implies that $x\in C$, so by disjunction elimination,
      $x\in C$.
    \item This second part of the theorem can be proved in a similar
      way to the first part, but with everything reversed.  We would
      replace $\supseteq$ with $\subseteq$, $\cup$ with $\cap$, etc.
  \end{enumerate}
\end{proof}

This theorem means that it would have been possible to define
intersection and union just in terms of the subset relation, without
any recourse to logic (though in this scenario, the subset relation
itself would still be defined in terms of logic).

We also have a theorem that goes the other way, which would allow us
(if we wanted) to define the subset relation just in terms of either
union or intersection, with no recourse to logic (but in this case,
union or intersection would have to be defined with logic).  Here it
is:

\begin{thm}\label{thm:relfromop}
  For sets $A$ and $B$,
  \begin{enumerate}[label=\alph*)]
  \item $A\subseteq B$ iff $A\cup B=B$
  \item $A\subseteq B$ iff $A=A\cap B$
  \end{enumerate}
\end{thm}
\begin{proof}
  \begin{enumerate}[label=\alph*)]
    \item We will prove each direction of the biconditional
      separately.
      \begin{description}
        \item[$\Rightarrow$)] Suppose $A\subseteq B$.  We prove $A\cup
          B=B$ by proving that $A\cup B \supseteq B$ and $A\cup
          B\subseteq B$.  If $x\in B$, then $x\in A\vee x\in B$ by
          disjunction introduction, so $x\in A\cup B$.  Thus $A\cup
          B\supseteq B$.  And if $x\in A\cup B$, then either $x\in A$
          or $x\in B$.  In the first case, $x\in B$ by our
          supposition.  In the second case, $x\in B$ by reiteration.
          Thus $x\in B$.  So $A\cup B\subseteq B$.
        \item[$\Leftarrow$)] Suppose $A\cup B=B$.  We will prove that
          $A\subseteq B$ using the definition of $\subseteq$.  Suppose
          $x\in A$.  Then $x\in A\vee x\in B$, so $x\in A\cup B$.  By
          our supposition, $x\in B$.
      \end{description}
    \item The second part is similar to the first.
  \end{enumerate}
\end{proof}

\section{DeMorgan's Laws}

So far we have seen the set-theoretic analogues of many laws of logic.
How about DeMorgan's Laws?  Well, we need an analogue of negation
first for that.

\begin{dfn}
  For a set $A$, define its complement $\overline{A}=\{x|x\not\in A\}$.
\end{dfn}

The complement of $A$ is the set of everything not in $A$.  It is the
set induced by the negation of the predicate which induces $A$.

Now we can state the set-theoretic analogue of DeMorgan's laws:

\begin{thm}
  For sets $A$ and $B$,
  \begin{enumerate}[label=\alph*)]
  \item $\overline{A\cup B}=\overline{A}\cap\overline{B}$
  \item $\overline{A\cap B}=\overline{A}\cup\overline{B}$
  \end{enumerate}
\end{thm}

Recall that a lot of our proofs above had two parts with similar
proofs.  DeMorgan's Law often let's us prove one of these parts from
the other.  For example, let's say that we know that union is
distributive over intersection (as proved above), but not \textit{vice
  versa} (as omitted above).  We can prove the one from the other:

\begin{proof}
  We know from the partial proof of Thm. \ref{thm:op} that for sets
  $A$, $B$, and $C$, $A\cup(B\cap C)=(A\cup B)\cap(A\cup C)$.  Now we
  substitute $\overline{A}$ for $A$, $\overline{B}$ for $B$, and
  $\overline{C}$ for $C$ to obtain $\overline{A}\cup(\overline{B}\cap
  \overline{C})=(\overline{A}\cup \overline{B})\cap(\overline{A}\cup
  \overline{C})$.  Then by DeMorgan's Laws,
  \begin{align*}
    \overline{A}\cup(\overline{B}\cap \overline{C}) &=
    (\overline{A}\cup\overline{B})\cap(\overline{A}\cup\overline{C})
    \\
    \overline{A}\cup\overline{B\cup C}&=\overline{A\cap
      B}\cap\overline{A\cap C} \\
    \overline{A\cap(B\cup C)}&=\overline{(A\cap B)\cup(A\cap C)}\\
    \overline{\overline{A\cap(B\cup C)}}&=\overline{\overline{(A\cap B)\cup(A\cap C)}}\\
    A\cap(B\cup C)&=(A\cap B)\cup(A\cap C)
  \end{align*}
  In the second-to-last step, we took the complement of both sides,
  and then in the last step, we used the fact that
  $\overline{\overline{S}}=S$.  This fact can be proved from the
  corresponding logical law $\oldneg\oldneg P \Leftrightarrow P$.
\end{proof}

\end{document}
